\usepackage{enumitem}
\usepackage{libertine}
\usepackage{changepage}
\usepackage{ifthen}
\usepackage{csquotes}
\usepackage{mdframed}
\usepackage{xcolor}
\usepackage{tcolorbox}
\usepackage{marginnote}
\usepackage{ragged2e}
\usepackage{pdfpages}



\setcounter{secnumdepth}{3}



\usepackage{geometry}
\geometry{
    top=1in,
    bottom=1in,
    inner=1in,
    outer=1in,
    heightrounded
}
\usepackage{fancyhdr}
\setlength{\headheight}{14.5pt}
\pagestyle{fancy}
\fancyhf{}
\fancyhead[LE,RO]{\thepage}
\fancyhead[RE]{\textbf{\nouppercase{\leftmark}}}
\fancyhead[LO]{\textbf{\nouppercase{\rightmark}}}
\fancyfoot[CE,CO]{}



\usepackage{bm}
\usepackage{bbm}
\usepackage{amsthm}
\usepackage{amsmath}
\usepackage{amssymb}
\usepackage{amsfonts}
\usepackage{mathrsfs}
\usepackage{esvect}
\usepackage{esint}
\usepackage{esdiff}
\usepackage{tikz}
\usepackage{graphicx}
\usepackage{siunitx}
\sisetup{
    product-symbol = \cdot,
    inter-unit-product = \cdot,
    exponent-product = \cdot,
    per-mode = power
}



\newtcolorbox{important}[1][]{colback=gray!20, colframe=gray!20, width=\textwidth, boxrule=0pt, leftrule=1mm, rightrule=1mm, arc=0mm, auto outer arc, #1}

\DeclareQuoteStyle{english}{\guillemotleft}{\guillemotright}{\textquoteleft}{\textquoteright}
\setquotestyle{english}

% environment for the quote
\newenvironment{note}{%
    \begin{adjustwidth}{1cm}{0cm}\noindent\ignorespaces
        }{%
    \end{adjustwidth}%
}

% environment for the quote with italic text
\newenvironment{inote}{%
    \begin{adjustwidth}{1cm}{0cm}\itshape\noindent\ignorespaces
        }{%
    \end{adjustwidth}%
}

% environment for the italic quote
\newenvironment{conclusion}{%
    \begin{adjustwidth}{1cm}{0cm}\itshape\noindent\ignorespaces\og{}\ignorespaces
        }{%
        \fg{}\end{adjustwidth}%
}

% environment for numbering example
\newcounter{myexample}[chapter]
\renewcommand{\themyexample}{Example \thechapter.\arabic{myexample}}
\makeatletter
\newenvironment{example}[1][]{%
    \refstepcounter{myexample}
    \begin{description}
        \item[\themyexample\space\ifx\relax#1\relax\else(#1) \fi:]
                \small %
            }{%
    \end{description}%
}
\makeatother

% environment for numbering counterexample
\newcounter{mycounterexample}[chapter]
\renewcommand{\themycounterexample}{Counterexample \thechapter.\arabic{mycounterexample}}
\makeatletter
\newenvironment{counterexample}[1][]{%
    \refstepcounter{mycounterexample}%
    \begin{description}%
        \item[\themycounterexample\space\ifx\relax#1\relax\else(#1) \fi:]%
            }{%
    \end{description}%
}
\makeatother

% environment for numbering remark
\newcounter{myremark}[chapter]
\renewcommand{\themyremark}{Remark \thechapter.\arabic{myremark}}
\makeatletter
\newenvironment{remark}[1][]{
    \refstepcounter{myremark}
    \noindent\ignorespaces
    \begin{description}
        \item[\themyremark\space\ifx\relax#1\relax\else(#1) \fi:]
            }{
    \end{description}%
}
\makeatother

% environment for numbering comment
\newcounter{mycomment}[chapter]
\renewcommand{\themycomment}{Comment \thechapter.\arabic{mycomment}}
\makeatletter
\renewenvironment{comment}[1][]{
    \refstepcounter{mycomment}
    \noindent\ignorespaces
    \begin{description}
        \item[\themycomment\space\ifx\relax#1\relax\else(#1) \fi:]
            }{
    \end{description}%
}
\makeatother



% MATHEMATICAL PART
% Define the custom theorem style
\newtheoremstyle{bfnote}
{}{}%                     % space above and below
{\itshape}%               % body font
{}%                       % indent amount
{\bfseries}%              % theorem head font
{.}%                      % punctuation after theorem head
{ }%                      % space after theorem head
{\thmname{#1}\thmnumber{ #2}\thmnote{ (#3)}}% theorem head spec

\theoremstyle{bfnote}

% define theorem environments with gray background
\newenvironment{definition}[1][]
{\begin{important}\begin{definitioninner}[#1]\ignorespaces}
            {\end{definitioninner}\end{important}}

\newenvironment{axiom}[1][]
{\begin{important}\begin{axiominner}[#1]\ignorespaces}
            {\end{axiominner}\end{important}}

\newenvironment{principle}[1][]
{\begin{important}\begin{principleinner}[#1]\ignorespaces}
            {\end{principleinner}\end{important}}

\newenvironment{law}[1][]
{\begin{important}\begin{lawinner}[#1]\ignorespaces}
            {\end{lawinner}\end{important}}

\newenvironment{lemma}[1][]
{\begin{important}\begin{lemmainner}[#1]\ignorespaces}
            {\end{lemmainner}\end{important}}

\newenvironment{proposition}[1][]
{\begin{important}\begin{propositioninner}[#1]\ignorespaces}
            {\end{propositioninner}\end{important}}

\newenvironment{theorem}[1][]
{\begin{important}\begin{theoreminner}[#1]\ignorespaces}
            {\end{theoreminner}\end{important}}

\newenvironment{corollary}[1][]
{\begin{important}\begin{corollaryinner}[#1]\ignorespaces}
            {\end{corollaryinner}\end{important}}

\newenvironment{postulation}[1][]
{\begin{important}\begin{postulationinner}[#1]\ignorespaces}
			{\end{postulationinner}\end{important}}

% define inner theorem environments
\newtheorem{definitioninner}{Definition}[chapter]
\newtheorem{axiominner}{Axiom}[chapter]
\newtheorem{principleinner}{Principle}[chapter]
\newtheorem{lawinner}{Law}[chapter]
\newtheorem{lemmainner}{Lemma}[chapter]
\newtheorem{propositioninner}{Proposition}[chapter]
\newtheorem{theoreminner}{Theorem}[chapter]
\newtheorem{corollaryinner}{Corollary}[theoreminner]
\newtheorem{exercise}{Excercise}[chapter]
\newtheorem{solution}{Solution}[chapter]
\newtheorem{postulationinner}{Postulation}[chapter]


% set notation
\newcommand{\N}{\mathbb{N}}
\newcommand{\Z}{\mathbb{Z}}
\newcommand{\Q}{\mathbb{Q}}
\newcommand{\I}{\mathbb{I}}
\newcommand{\R}{\mathbb{R}}
\newcommand{\C}{\mathbb{C}}
\newcommand{\E}{\mathbb{E}}
\renewcommand{\P}{\mathbb{P}}
\newcommand{\Id}{\mathbbm{1}}
\let\emptyset\varnothing

\DeclareMathOperator{\tr}{tr}
\DeclareMathOperator{\Idm}{Id}



% long arc notation for time derivative
\makeatletter
\DeclareFontFamily{U}{tipa}{}
\DeclareFontShape{U}{tipa}{m}{n}{<->tipa10}{}
\newcommand{\arc@char}{{\usefont{U}{tipa}{m}{n}\symbol{62}}}%
\newcommand{\arc}[1]{\mathpalette\arc@arc{#1}}
\newcommand{\arc@arc}[2]{%
    \sbox0{$\m@th#1#2$}%
    \vbox{
        \hbox{\resizebox{\wd0}{\height}{\arc@char}}
        \nointerlineskip
        \box0
    }%
}
\makeatother

\newcommand{\difft}[1]{\dot{\arc{#1}}}
\newcommand{\interior}[1]{\mathring{#1}}



\usepackage{xparse}

\newcounter{article}
\NewDocumentEnvironment{article}{o}{%
  \refstepcounter{article}%
  \noindent$\blacktriangleright$~\textbf{\thearticle}%
  \IfValueT{#1}{.~\textbf{#1}}.~
  \ignorespaces
}{%
  \par\medskip
}



% HYPERLINK PART
\usepackage[hidelinks]{hyperref}